% LaTeX template for creating an MNRAS paper
%
% v3.2 released 20 July 2023
% (version numbers match those of mnras.cls)
%
% Copyright (C) Royal Astronomical Society 2015
% Authors:
% Keith T. Smith (Royal Astronomical Society)

%%%%%%%%%%%%%%%%%%%%%%%%%%%%%%%%%%%%%%%%%%%%%%%%%%%%%%%%%%%%%%%%%%%%%%%%%%%%%%%%%%%%%%%%%%%%%%%%%%%%
% Basic setup. Most papers should leave these options alone.
\documentclass[fleqn,usenatbib]{mnras}

% MNRAS is set in Times font. If you don't have this installed (most LaTeX
% installations will be fine) or prefer the old Computer Modern fonts, comment
% out the following line
\usepackage{newtxtext,newtxmath}
% Depending on your LaTeX fonts installation, you might get better results with one of these:
%\usepackage{mathptmx}
%\usepackage{txfonts}

% Use vector fonts, so it zooms properly in on-screen viewing software
% Don't change these lines unless you know what you are doing
\usepackage[T1]{fontenc}

% Allow "Thomas van Noord" and "Simon de Laguarde" and alike to be sorted by "N" and "L" etc. in the bibliography.
% Write the name in the bibliography as "\VAN{Noord}{Van}{van} Noord, Thomas"
\DeclareRobustCommand{\VAN}[3]{#2}
\let\VANthebibliography\thebibliography
\def\thebibliography{\DeclareRobustCommand{\VAN}[3]{##3}\VANthebibliography}


%%%%% AUTHORS - PLACE YOUR OWN PACKAGES HERE %%%%%

% Only include extra packages if you really need them. Avoid using amssymb if newtxmath is enabled, as these packages can cause conflicts. newtxmatch covers the same math symbols while producing a consistent Times New Roman font. Common packages are:
\usepackage{graphicx}	
\usepackage{amsmath}	
\usepackage{physics}
\usepackage{enumitem}
\usepackage{anyfontsize}
\usepackage{placeins}
\usepackage{cleveref}


%%%%%%%%%%%%%%%%%%%%%%%%%%%%%%%%%%%%%%%%%%%%%%%%%%%%%%%%%%%%%%%%%%%%%%%%%%%%%%%%%%%%%%%%%%%%%%%%%%%%

%%%%% AUTHORS - PLACE YOUR OWN COMMANDS HERE %%%%%

% Please keep new commands to a minimum, and use \newcommand not \def to avoid
% overwriting existing commands. Example:
%\newcommand{\pcm}{\,cm$^{-2}$}	% per cm-squared

\graphicspath{{images/}}
%\raggedbottom               %This will let the height of the textblock vary from page to page. 

% Command to use placeins with \subsection
\makeatletter
\AtBeginDocument{%
  \expandafter\renewcommand\expandafter\subsection\expandafter
    {\expandafter\@fb@secFB\subsection}%
  \newcommand\@fb@secFB{\FloatBarrier
    \gdef\@fb@afterHHook{\@fb@topbarrier \gdef\@fb@afterHHook{}}}%
  \g@addto@macro\@afterheading{\@fb@afterHHook}%
  \gdef\@fb@afterHHook{}%
}
\makeatother

%%%%%%%%%%%%%%%%%%%%%%%%%%%%%%%%%%%%%%%%%%%%%%%%%%%%%%%%%%%%%%%%%%%%%%%%%%%%%%%%%%%%%%%%%%%%%%%%%%%%


%%%%%%%%%%%%%%%%%%% TITLE PAGE %%%%%%%%%%%%%%%%%%%%%%%%%%%%%%%%%%%%%%%%%%%%%%%%%%%%%%%%%%%%%%%%%%%%%

% Title of the paper, and the short title which is used in the headers.
% Keep the title short and informative.
\title[]{Simulating the formation of supermassive black hole binaries}

% The list of authors, and the short list which is used in the headers.
% If you need two or more lines of authors, add an extra line using \newauthor
\author[Marco Bianchi]{
Marco Bianchi
\\
% List of institutions
University of Milano-Bicocca, Physics Department, Astrophysics and Space Physics master degree\\
Dynamics of Stellar Systems 2023-2024
}

% Enter the current year, for the copyright statements etc.
\pubyear{2025}

% Don't change these lines
\begin{document}
\label{firstpage}
\pagerange{\pageref{firstpage}--\pageref{lastpage}}
\maketitle

% Abstract of the paper
\begin{abstract}
  \large The Cocoon Nebula (IC 5146) is an HII region embedded within the Barnard 168 absorption nebula, making it a prime target for studying dust extinction. 
  We obtained photometric images of IC 5146 using H$\alpha$ and H$\beta$ filters and analyzed the Balmer decrement to create a two-dimensional map of the color excess E(B-V), which serves as a proxy for dust column density. 
  Our results indicate a median E(B-V) of $\approx 0.461 \pm 0.008$ across the nebula, with a standard deviation of $\approx 0.34$. 
  A comparison with previous spectroscopic studies reveals a discrepancy in E(B-V) estimates, which we attribute to methodological differences in spatial sampling.\\
  Furthermore, given the nebula's nearly spherical morphology and its ionization by a single B0.5V-type star (BD+46 3474), we tested the Strömgren sphere model. 
  By incorporating literature values for electron density and temperature, as well as for the radius and temperature of BD+46 3474, we derived a Strömgren radius of $\approx 1.14$ pc, which is broadly consistent with the observed nebular size. 
  However, uncertainties in the nebula's distance and simplifying assumptions in the Strömgren model highlight the need for more detailed modeling.
\end{abstract} 
%%%%%%%%%%%%%%%%%%%%%%%%%%%%%%%%%%%%%%%%%%%%%%%%%%%%%%%%%%%%%%%%%%%%%%%%%%%%%%%%%%%%%%%%%%%%%%%%%%%%




%%%%%%%%%%%%%%%%% BODY OF PAPER %%%%%%%%%%%%%%%%%%%%%%%%%%%%%%%%%%%%%%%%%%%%%%%%%%%%%%%%%%%%%%%%%%%%

\section{Introduction}\label{sec:introduction}
It is a well-established fact that most galaxies host a massive black hole (MBH) at their center, with masses ranging from a few million to several billion solar masses. In the latter case, these are referred to as supermassive black holes (SMBHs).
On the other hand, galactic mergers are expected to be frequent events in the Universe, and indeed we observe many galaxies in various stages of the merging process.
This raises the question of what happens to MBHs during these mergers.
The most likely outcome is the formation of a MBH binary at the center of the newly formed galaxy.
The process leading to the formation and the evolution of such a bound state is complex and involves several stages, which were originally outlined by \cite{Begelman1980}.

\subsection{Dynamical friction}\label{sec:introduction_dynamical_friction}
In this work, we focus on the process that causes the MBHs to sink towards the core of the merger remnant, a mechanism known as \textbf{dynamical friction}.
The main idea is that a massive object moving through a stellar system will transfer energy and angular momentum to the surrounding stars, slowing down and thus spiraling inward, toward the center of mass of the system. Although the object may gain speed as it moves into a region with deeper gravitational potential, this does not contradic the previous statement, since the total orbital energy is still decreasing.
The mathematical formalization of dynamical friction was first proposed by \cite{Chandrasekhar1943}, based on the assumption that the stellar background is homogeneous and isotropic.
In particular, the perturber is subject to a deceleration given by the following equation:
{\fontsize{8.5pt}{8.5pt}\begin{equation}
    \dfrac{d\vec{v}_M}{dt} \, = \, -16 \pi^2 G^2 m \left(m+M\right) \ln \Lambda \left[\int^{v_M}_0 f(v_m) v_m^2 dv_m\right] \dfrac{\vec{v}_M}{v_M^3},
    \label{eq:chandrasekhar}
\end{equation}}
where $\ln \Lambda$ is the Coulomb logarithm (which depends on the maximum and minimum impact parameters of the interaction), $m$ is the mass of the background particles, $M$ is the mass of the perturber, and $f(v_m)$ is the distribution function of the background particles.
Dynamical friction is most efficient when $v_M \simeq v_m$, which is typically the case during the inspiral, and becomes negligible for $v_M \ll v_m$ or $v_M \gg v_m$.

Once the MBHs have sunk to the center of the merger remnant and their separation is such that each is primarily influenced by the gravitational pull of the other, they form a bound state.
The critical separation at which this occurs can be approximated by the \textbf{influence radius}, i.e., the distance at which the gravitational potential energy of the MBH is comparable to the kinetic energy of the surrounding stars:
{\fontsize{7.4pt}{7.4pt}\begin{equation}
    r_\text{inf} = \dfrac{GM}{\sigma^2} 
    \approx 1 \left(\dfrac{M}{10^6 M_\odot}\right) \left(\dfrac{\sigma}{70 \text{km/s}}\right)^{-2} \text{pc} 
    \approx 50 \left(\dfrac{M}{10^9 M_\odot}\right) \left(\dfrac{\sigma}{300 \text{km/s}}\right)^{-2} \text{pc} ,
    \label{eq:influence_radius}
\end{equation}}
where $\sigma$ is the velocity dispersion of the stars surrounding the MBH.

The formation of the binary causes a sudden change in the velocities of the MBHs relative to the surrounding stars, making dynamical friction inefficient.
Morever, once the MBHs are bound, dynamical friction acts of the center of mass of the binary, rather than on the individual MBHs.
The next step is to determine whether other mechanisms can reduce the binary separation further, eventually leading to coalescence.
It is well known that compact-object binaries lose energy and angular momentum via gravitational waves emission, thus shrinking their orbits.
However, this process becomes efficient only when the binary separation is already very small, of the order of a few milliparsecs.
We therefore need to identify a mechanism capable of bridging the gap from parsecs to milliparsecs.
This is the so-called \textbf{final-parsec problem}, which has been the subject of extensive research in the past decades.
The most promising solution is \textbf{stellar hardening}, a process in which stars undergo three-body interactions with the binary, extracting energy and angular momentum from it.
%%%%%%%%%%%%%%%%%%%%%%%%%%%%%%%%%%%%%%%%%%%%%%%%%%

\subsection{Modeling galactic bulges with Barnes' treecode}\label{sec:introduction_models}
The efficiency of dynamical friction depends on the density and velocity distribution of the stars which constitute the bulge of the merger remnant.
This is typically modeled as a spherically-symmetric distribution of particles, with a density profile that decays with distance from the center.
The model most commonly used in analytical calculations is the \textbf{singular isothermal sphere} (SIS), whose density and mass profiles are given by:
\begin{equation}
    \rho_\text{SIS}(r) = \dfrac{\sigma^2}{2 \pi G r^2} \:\: , \:\: M_\text{SIS}(r) = \dfrac{2 \sigma^2}{G} r \:\:.
    \label{eq:sis_density}
\end{equation}
However, this model is not realistic, as it diverges at $r=0$ and predicts an infinite total mass.
In the next section, we will introduce more physical alternatives, which are still spherically symmetric but have finite total mass.

In any case, modeling a galactic bulge requires a large number of particles, typically of the order of $10^6$ or more, and thus the use of direct N-body simulations — which have complexity $\mathcal{O}(N^2)$ — is not feasible.
Instead, we will use the \textbf{treecode} algorithm by \cite{Barnes1986}, which is a method that approximates the gravitational potential of a system of particles by grouping them into a hierarchical tree structure.
%%%%%%%%%%%%%%%%%%%%%%%%%%%%%%%%%%%%%%%%%%%%%%%%%%
%%%%%%%%%%%%%%%%%%%%%%%%%%%%%%%%%%%%%%%%%%%%%%%%%%%%%%%%%%%%%%%%%%%%%%%%%%%%%%%%%%%%%%%%%%%%%%%%%%%%


\section{Distributing particles according to various models}\label{sec:observation}
\subsection{Isothermal sphere}\label{sec:isothermal_sphere}
prova
%%%%%%%%%%%%%%%%%%%%%%%%%%%%%%%%%%%%%%%%%%%%%%%%%%

\subsection{King model}\label{sec:King_model}
prova
%%%%%%%%%%%%%%%%%%%%%%%%%%%%%%%%%%%%%%%%%%%%%%%%%%

\subsection{Hernquist model}\label{sec:Hernquist_model}
prova
%%%%%%%%%%%%%%%%%%%%%%%%%%%%%%%%%%%%%%%%%%%%%%%%%%
%%%%%%%%%%%%%%%%%%%%%%%%%%%%%%%%%%%%%%%%%%%%%%%%%%%%%%%%%%%%%%%%%%%%%%%%%%%%%%%%%%%%%%%%%%%%%%%%%%%%


\section{Simulation}\label{sec:simulation}
%%%%%%%%%%%%%%%%%%%%%%%%%%%%%%%%%%%%%%%%%%%%%%%%%%
%%%%%%%%%%%%%%%%%%%%%%%%%%%%%%%%%%%%%%%%%%%%%%%%%%%%%%%%%%%%%%%%%%%%%%%%%%%%%%%%%%%%%%%%%%%%%%%%%%%%



\section{Conclusion}\label{sec:conclusion}
%%%%%%%%%%%%%%%%%%%%%%%%%%%%%%%%%%%%%%%%%%%%%%%%%%
%%%%%%%%%%%%%%%%%%%%%%%%%%%%%%%%%%%%%%%%%%%%%%%%%%%%%%%%%%%%%%%%%%%%%%%%%%%%%%%%%%%%%%%%%%%%%%%%%%%%


%%%%%%%%%%%%%%%%%%%% REFERENCES %%%%%%%%%%%%%%%%%%%%%%%%%%%%%%%%%%%%%%%%%%%%%%%%%%%%%%%%%%%%%%%%%%%%
% The best way to enter references is to use BibTeX:
\bibliographystyle{mnras}
\bibliography{bibliography} 
%%%%%%%%%%%%%%%%%%%%%%%%%%%%%%%%%%%%%%%%%%%%%%%%%%%%%%%%%%%%%%%%%%%%%%%%%%%%%%%%%%%%%%%%%%%%%%%%%%%%


% Don't change these lines
\label{lastpage}
\end{document}
